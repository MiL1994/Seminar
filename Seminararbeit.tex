% This is "sig-alternate.tex" V2.1 April 2013
% This file should be compiled with V2.5 of "sig-alternate.cls" May 2012
%
% This example file demonstrates the use of the 'sig-alternate.cls'
% V2.5 LaTeX2e document class file. It is for those submitting
% articles to ACM Conference Proceedings WHO DO NOT WISH TO
% STRICTLY ADHERE TO THE SIGS (PUBS-BOARD-ENDORSED) STYLE.
% The 'sig-alternate.cls' file will produce a similar-looking,
% albeit, 'tighter' paper resulting in, invariably, fewer pages.
%
% ----------------------------------------------------------------------------------------------------------------
% This .tex file (and associated .cls V2.5) produces:
%       1) The Permission Statement
%       2) The Conference (location) Info information
%       3) The Copyright Line with ACM data
%       4) NO page numbers
%
% as against the acm_proc_article-sp.cls file which
% DOES NOT produce 1) thru' 3) above.
%
% Using 'sig-alternate.cls' you have control, however, from within
% the source .tex file, over both the CopyrightYear
% (defaulted to 200X) and the ACM Copyright Data
% (defaulted to X-XXXXX-XX-X/XX/XX).
% e.g.
% \CopyrightYear{2007} will cause 2007 to appear in the copyright line.
% \crdata{0-12345-67-8/90/12} will cause 0-12345-67-8/90/12 to appear in the copyright line.
%
% ---------------------------------------------------------------------------------------------------------------
% This .tex source is an example which *does* use
% the .bib file (from which the .bbl file % is produced).
% REMEMBER HOWEVER: After having produced the .bbl file,
% and prior to final submission, you *NEED* to 'insert'
% your .bbl file into your source .tex file so as to provide
% ONE 'self-contained' source file.
%
% ================= IF YOU HAVE QUESTIONS =======================
% Questions regarding the SIGS styles, SIGS policies and
% procedures, Conferences etc. should be sent to
% Adrienne Griscti (griscti@acm.org)
%
% Technical questions _only_ to
% Gerald Murray (murray@hq.acm.org)
% ===============================================================
%
% For tracking purposes - this is V2.0 - May 2012

\documentclass{sig-alternate-05-2015}
\usepackage[utf8]{inputenc}

\begin{document}

% Copyright
\setcopyright{acmcopyright}
%\setcopyright{acmlicensed}
%\setcopyright{rightsretained}
%\setcopyright{usgov}
%\setcopyright{usgovmixed}
%\setcopyright{cagov}
%\setcopyright{cagovmixed}
%
% --- Author Metadata here ---
\conferenceinfo{WOODSTOCK}{'97 El Paso, Texas USA}
%\CopyrightYear{2007} % Allows default copyright year (20XX) to be over-ridden - IF NEED BE.
%\crdata{0-12345-67-8/90/01}  % Allows default copyright data (0-89791-88-6/97/05) to be over-ridden - IF NEED BE.
% --- End of Author Metadata ---

\title{Alternate {\ttlit ACM} SIG Proceedings Paper in LaTeX
Format\titlenote{(Produces the permission block, and
copyright information). For use with
SIG-ALTERNATE.CLS. Supported by ACM.}}
\subtitle{[Extended Abstract]
\titlenote{A full version of this paper is available as
\textit{Author's Guide to Preparing ACM SIG Proceedings Using
\LaTeX$2_\epsilon$\ and BibTeX} at
\texttt{www.acm.org/eaddress.htm}}}

\numberofauthors{1} %  in this sample file, there are a *total*
% of EIGHT authors. SIX appear on the 'first-page' (for formatting
% reasons) and the remaining two appear in the \additionalauthors section.
%
\author{
% You can go ahead and credit any number of authors here,
% e.g. one 'row of three' or two rows (consisting of one row of three
% and a second row of one, two or three).
%
% The command \alignauthor (no curly braces needed) should
% precede each author name, affiliation/snail-mail address and
% e-mail address. Additionally, tag each line of
% affiliation/address with \affaddr, and tag the
% e-mail address with \email.
%
% 1st. author
\alignauthor
Maurice Liebelt\titlenote\\
       \affaddr{Institute for Clarity in Documentation}\\
       \affaddr{1932 Wallamaloo Lane}\\
       \affaddr{Wallamaloo, New Zealand}\\
       \email{trovato@corporation.com}
}
% There's nothing stopping you putting the seventh, eighth, etc.
% author on the opening page (as the 'third row') but we ask,
% for aesthetic reasons that you place these 'additional authors'
% in the \additional authors block, viz.
\additionalauthors{Additional authors: John Smith (The Th{\o}rv{\"a}ld Group,
email: {\texttt{jsmith@affiliation.org}}) and Julius P.~Kumquat
(The Kumquat Consortium, email: {\texttt{jpkumquat@consortium.net}}).}
\date{30 July 1999}
% Just remember to make sure that the TOTAL number of authors
% is the number that will appear on the first page PLUS the
% number that will appear in the \additionalauthors section.

\maketitle
\begin{abstract}
\end{abstract}


%
% The code below should be generated by the tool at
% http://dl.acm.org/ccs.cfm
% Please copy and paste the code instead of the example below. 
%
\begin{CCSXML}
<ccs2012>
 <concept>
  <concept_id>10010520.10010553.10010562</concept_id>
  <concept_desc>Computer systems organization~Embedded systems</concept_desc>
  <concept_significance>500</concept_significance>
 </concept>
 <concept>
  <concept_id>10010520.10010575.10010755</concept_id>
  <concept_desc>Computer systems organization~Redundancy</concept_desc>
  <concept_significance>300</concept_significance>
 </concept>
 <concept>
  <concept_id>10010520.10010553.10010554</concept_id>
  <concept_desc>Computer systems organization~Robotics</concept_desc>
  <concept_significance>100</concept_significance>
 </concept>
 <concept>
  <concept_id>10003033.10003083.10003095</concept_id>
  <concept_desc>Networks~Network reliability</concept_desc>
  <concept_significance>100</concept_significance>
 </concept>
</ccs2012>  
\end{CCSXML}

\ccsdesc[500]{Computer systems organization~Embedded systems}
\ccsdesc[300]{Computer systems organization~Redundancy}
\ccsdesc{Computer systems organization~Robotics}
\ccsdesc[100]{Networks~Network reliability}


%
% End generated code
%

%
%  Use this command to print the description
%
\printccsdesc

% We no longer use \terms command
%\terms{Theory}

\keywords{ACM proceedings; \LaTeX; text tagging}

\section{Einleitung}
\section{Hintergrund}
\section{Realisierung von DSPL}
Im folgenden Abschnitt werden unterschiedliche Ansätze vorgestellt, mit denen es möglich ist dynamische Software Produktlinien sowohl zu modellieren, als auch zu implementieren. 
Zunächst wird ein Modellierungsansatz vorgestellt, der es ermöglichen soll, die dynamischen Features und deren Abhängigkeiten zu anderen statischen oder dynamischen Features abzubilden. 
Dieser Ansatz basiert auf der Odyssey-Fex Notation und erweitert diese um kontextspezifische Erweiterungen. 
Dabei wird allerdings nur auf die Modellierung eines kontextbewussten Systems eingegangen, da die Implementierung dahinter unabhängig von der gewählten Modellierungssprache stattfindet. 
Um im gleichen Zug eine anderen Herangehensweise vorzustellen, bei der der Modellierungsansatz sehr spezfisch auf die eigentliche Implementierung ausgelegt ist. 
Dabei wird der Umstand genutzt, dass ein Aspekt in aspektorientierten Programmiersprachen als Feature eines Feature-Modells einer SPL implementiert werden kann. 
Ein weiterer Ansatz zur Realisierung von DSPLs ist die Implementierung mittels kontextorientierter Programmierung. 
Diese soll besonders im Cloud SaaS Bereich Anwendung finden. (Feature Clouds Programming) \\

\subsection{Modellierung}

\subsubsection{UbiFEX}
Was heißt Ubifex?
Der nachfolgdend beschriebene Ansatz beschreibt eine Erweiterung der Odyssey-Fex Notation. 
In der Standardnotation können verschiedene Elementa dargestellt werden. 
Das können unter Anderem konzeptionelle, funktionale oder technologische Features sein. 
Innerhalb der Modellierung kann ein Feature entweder ein „variation point“, eine Variante oder eine Invariante. 
Die Beziehungen zwischen einzelnen Features wird mittels der UML- Beziehungen bei Klassen dargestellt. 
Hinzu kommen explizite include und exclude Beziehung zwischen Variations points. \\

HIER EIN EIGENES BEISPIEL DER OdysseyFEX NOTATION

Zusätzliche können bei varaition points in FIG1 definieren, wie viele Varianten in an diesem Punkt eingebunden werden können oder , ob sie sich gegenseitig ausschließen. 
Um in dieser Notation nun Komtextinformationen abbilden zu können, werden sogenannte Context-entity features  modelliert, die verschiedenen Kontextabhängige Situtationen beschreiben. 
Das können unter anderem Sachen, Personen oder Orte sein. 
Damit die Kontexte dieser Kontexrt entitäten adäquat beschrieben werden können, werden Kontext information features definiert. 
Diese beschreiben die für den Kontext relevanten Daten. Diese bestehen unter anderem aus den in \ref{tab:context} beschriebenen Informationen

\begin{table}
\label{tab:context}
\centering
\caption{Kontextbeschreibung}
\begin{tabular}{|l|l|}
\hline
Name &	Name der Kontextinformation \\ \hline
Beschreibung	& Beschreibung der Kontext \\ \hline
Typ &	Datentyp der Information \\ \hline
Persistenz &	Statische oder dynamische Information \\ \hline
\end{tabular}
\end{table}



Kontext entitäten können, wie Features in dieser Noataion, mittels UML-Relationen in Beziehung in stehen. 
Da diese Kontext Entitäten und Kontext Informationen allerdings noch nicht dazu geeignet sind dynamische Zusammenhänge zu repräsentieren. 
Aus diesem Grund werden zusätzliche Kontextdefinitionen eingeführt. 
Diese beschreiben die für die modellierte Domain Zustände. Genauer gesagt, bei welchen Kontextinformationswerten der Kontextentität greift welche Kontextdefinition. 
Aus bestimmten Kontexten können nun dynamische Regeln für Featureinteraktion abgeleitet werden. 
Beispielsweise ist durch eine Änderung der Umgebung eine Kontextinformation geändert. 
Daraus folgt, dass sich das System nun in einem anderen definierten Kontext befindet. 
Tritt dieser Fall ein, so ist noch nicht klar, welche Änderungen sich im System ergeben. 
Um diesen Umstand aufzuklären, werden Kontextregeln definiert. 
Diese Regeln bestehen simpel aus einer Kontextdefinition, einer Implikation und einer daraus folgenden Konsequenz. 
Veranschaulicht heißt es, dass wenn Kontext A aktiv wird, dass Feature B hinzugefügt werden muss.
Das alles noch anhand eines Beispiels!

\subsubsection{Feature Cloud Model}
Das Feature Cloud Model ist eigentlich Sinne kein Ansatz für dynamische Software Produktlininen, sondern in erster Liniee ein Ansatz zu dynamische Adaption von Features zur Laufzeit. 
Doch gerade weil sich dieser Ansatz mit dynamischen Einbinden von Features in ein bestehendes System beschäftigt, ist ebenfalls zu betrachten, ob er nicht auch ein möglichen Ansatz zur Implementierung von dynamischen Software Produktlinien darstellt.\\ 
Grundlegend für diesen Implementierungsansatz ist das kontextorientiert Programmierparadiga. 
Dort werden grundlegende Dinge für den Kontext definiert. 
Es bietet die Möglichkeit die folgenden  MODULE zu implementieren, welche die Funktionalität für dynamische Adaption von Programmteilen oder Features ermöglich, nachstehend näher erläutert werden: \\
\begin{enumerate}
	\item Kontext Definition
	\item	Definition von Verhaltensadaptionen
	\item	Kontextauswahl
	\item	Kontextkomposition
	\item	Kontextabhängigkeiten
\end{enumerate}
Zum einen kann der eigentliche Kontext definiert ist. 
Weiterhin erlauben KOP Verhaltensadaptionen für jeden Kontext. 
Diese Verhaltensadaptionen überschreiben das eigentliche Standardverhalten eines Systems in einem bestimmten Feature.  
Diese Verhaltensadaptionen können auch als Gruppe von Methoden oder Funktionen aufgefasst werden. 
Von diesen Funktionen würde dann jeweils nur eine in einem oder mehreren entsprechenden Kontexten genutzt werden. 
Eine weiter Möglichkeit von KOP ist die Selektion von Kontexten entsprechend der aktuellen Umgebung. 
Des Weiteren 



\subsection{Citations}
Citations to articles \cite{bowman:reasoning,
clark:pct, braams:babel, herlihy:methodology},
conference proceedings \cite{clark:pct} or
books \cite{salas:calculus, Lamport:LaTeX} listed
in the Bibliography section of your
article will occur throughout the text of your article.
You should use BibTeX to automatically produce this bibliography;
you simply need to insert one of several citation commands with
a key of the item cited in the proper location in
the \texttt{.tex} file \cite{Lamport:LaTeX}.
The key is a short reference you invent to uniquely
identify each work; in this sample document, the key is
the first author's surname and a
word from the title.  This identifying key is included
with each item in the \texttt{.bib} file for your article.

The details of the construction of the \texttt{.bib} file
are beyond the scope of this sample document, but more
information can be found in the \textit{Author's Guide},
and exhaustive details in the \textit{\LaTeX\ User's
Guide}\cite{Lamport:LaTeX}.

This article shows only the plainest form
of the citation command, using \texttt{{\char'134}cite}.
This is what is stipulated in the SIGS style specifications.
No other citation format is endorsed or supported.



\begin{figure*}
\centering
\includegraphics{flies}
\caption{A sample black and white graphic
that needs to span two columns of text.}
\end{figure*}


\begin{figure}
\centering
\includegraphics[height=1in, width=1in]{rosette}
\caption{A sample black and white graphic that has
been resized with the \texttt{includegraphics} command.}
\vskip -6pt
\end{figure}


%
% The following two commands are all you need in the
% initial runs of your .tex file to
% produce the bibliography for the citations in your paper.
\bibliographystyle{abbrv}
\bibliography{sigproc}  % sigproc.bib is the name of the Bibliography in this case
% You must have a proper ".bib" file
%  and remember to run:
% latex bibtex latex latex
% to resolve all references
%
% ACM needs 'a single self-contained file'!
%
%APPENDICES are optional
%\balancecolumns
\appendix
%Appendix A
\section{Headings in Appendices}
The rules about hierarchical headings discussed above for
the body of the article are different in the appendices.
In the \textbf{appendix} environment, the command
\textbf{section} is used to
indicate the start of each Appendix, with alphabetic order
designation (i.e. the first is A, the second B, etc.) and
a title (if you include one).  So, if you need
hierarchical structure
\textit{within} an Appendix, start with \textbf{subsection} as the
highest level. Here is an outline of the body of this
document in Appendix-appropriate form:
\subsection{Introduction}
\subsection{The Body of the Paper}
\subsubsection{Type Changes and  Special Characters}
\subsubsection{Math Equations}
\paragraph{Inline (In-text) Equations}
\paragraph{Display Equations}
\subsubsection{Citations}
\subsubsection{Tables}
\subsubsection{Figures}
\subsubsection{Theorem-like Constructs}
\subsubsection*{A Caveat for the \TeX\ Expert}
\subsection{Conclusions}
\subsection{Acknowledgments}
\subsection{Additional Authors}
This section is inserted by \LaTeX; you do not insert it.
You just add the names and information in the
\texttt{{\char'134}additionalauthors} command at the start
of the document.
\subsection{References}
Generated by bibtex from your ~.bib file.  Run latex,
then bibtex, then latex twice (to resolve references)
to create the ~.bbl file.  Insert that ~.bbl file into
the .tex source file and comment out
the command \texttt{{\char'134}thebibliography}.
% This next section command marks the start of
% Appendix B, and does not continue the present hierarchy
\section{More Help for the Hardy}
The sig-alternate.cls file itself is chock-full of succinct
and helpful comments.  If you consider yourself a moderately
experienced to expert user of \LaTeX, you may find reading
it useful but please remember not to change it.
%\balancecolumns % GM June 2007
% That's all folks!
\end{document}
